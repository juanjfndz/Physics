% DUDA: Bert Repasar. Darle más forma de introducción 
% DUDA: Bert Relación entre espín y carácter tensorial del campo. Relación entre simetría gauge y carácter massless 
%		       Y las posibles igualdades y desigualdades de los campos mass y massless
% DUDA: Si después de una frase no hay una conexión, no es una oración. 
\chapter{Introducción} \label{Introduccion}

\justify{Michael Faraday (1791-1869) introdujo el concepto de \textit{campo} por primera vez en la historia de la física. En lugar de entender la interacción magnética como una acción a distancia, Faraday propuso la existencia de un mediador que transporta la información entre los objetos interactuantes. De esta forma, el emisor sufre una perturbación que afecta al campo de su entorno, después esta perturbación se transmite a lo largo de todo el campo y finalmente llega a los receptores.}

% DUDA: Citar el artículo de James Clerck Maxwell.
\justify{La teoría clásica de campos revoluciona la física del siglo XIX porque se emplea como paradigma para modelar las interacciones con campos. El ejemplo más claro es el electromagnetismo: en 1885 James Clerk Maxwell (1831-1879) publica un artículo en el que describe la dinámica del campo electromagnético. En segundo lugar, en ese mismo siglo se reformula la interacción gravitatoria de Newton a través de su campo homónimo.}

\justify{En 1905 se publica la teoría de la Relatividad Especial de Albert Einstein (1879-1955), esta nueva teoría sienta las bases de la física del siglo XX. De la Relatividad Especial también surge el concepto de spin y, con él, la teoría del campo de spin. La teoría del campo de spin se vuelve imprescindible para el desarrollo de la teoría Cuántica de Campos, teoría que explica el comportamiento de las partículas fundamentales hasta hoy conocidas.}

% DUDA: la gravedad no se encuentra descrita como un campo de spin-2
\justify{En física se clasifican los campos cuánticos en dos tipos: los que conforman la materia, que se denominan fermiónicos y presentan spin semi-entero; y los campos que median en las interacciones, llamados bosónicos y con spin entero. Todas las interacciones, excepto la gravedad, se encuentran bien descritas por campos bosónicos y forman parte del modelo Estándar de Partículas. Este modelo describe cómo funcionan las fuerzas y cómo interactúan todas las las partículas elementales, en él se encuentran los campos bosónicos de spin-0 \cite{HiggsParticle}, de spin-1 no masivos \cite{Aste:1998iw, Schwartz:2014sze} y spin-1 masivos \cite{WZParticle, Aste:1997rh}.}

\justify{La única interacción que no se encuentra actualmente en el modelo Estándar de Partículas es la gravedad. Esta interacción se describe por la teoría de la  Relatividad General, se trata de una teoría de la gravedad desarrollada por Albert Einstein y publicada 1915. La teoría explica que la fuerza de la gravedad se debe a la distorsión de las masas y la geometría del espacio-tiempo. La Relatividad General es una reformulación de la gravedad newtoniana que incluye la interacción gravitatoria en el marco de la Relatividad Especial. Aunque la Relatividad General no parece estar bajo una teoría de campo de spin, su régimen lineal se comporta como un campo de spin 2 (\cite{Wald:1984rg}, Apdo. 4.4;  \cite{JanssenBook}, Apdo. 28.4.), este resultado motiva la búsqueda de una teoría de la gravedad basada en campos de spin, y una posible gravedad cuántica.}

\justify{Las teorías de campos de spin proporcionan un marco útil para estudiar los conceptos más fundamentales de la física moderna, permiten entender cómo se comportan la mayoría de las interacciones fundamentales y estudiar qué peculiaridades presentan. Además, las teorías de campos sirven para investigar y buscar respuestas a las preguntas actuales de la física teórica. Algunos de los ejemplos más llamativas de nueva física son: la búsqueda de una partícula mediadora de la gravedad \cite{Wellmann:2001sx}, o la búsqueda de física más allá del modelo estándar, como las partículas candidatas a materia oscura \cite{Feng:2010gw, Cebrian:2022brv}; o la investigación sobre el origen de la energía oscura \cite{Tsujikawa:2013fta}...}

\justify{Esta tesis nos introduce en los fundamentos de la teoría de campos de spin, centrándose en el análisis de las teorías de campos bosónicos. Se propone como objetivo de este estudio el ganar experiencia y desarrollo de habilidades en la construcción de teorías de campos bosónicos, y comprender cómo se describen las actuales teorías de la física de interacciones fundamentales.}

\justify{En particular, la tesis se centra en el análisis de los grados de libertad de los campos asociados a las teorías, en el análisis de los grados de libertad aparentes y la búsqueda de ligaduras para converger al número de grados esperados por la teoría. De forma emergente también comprobaremos las diferencias y similitudes entre campos masivos y no masivos de mismo spin, y para el último caso también haremos hincapié en el análisis de sus grados de libertad y el comportamiento de spin asociado al campo.}

\justify{El trabajo comienza con el capítulo \ref{Preliminar}, en él se introducen tres aspectos relevantes para analizar las teorías de los capítulos siguientes: el teorema de Ostrogradsky, que permite identificar posibles comportamientos anómalos a las predicciones clásicas; la clasificación de Wigner, que nos establece una predicción de los grados de libertad de un campo de spin; y el dual de Hodge que se trata de un operador matemático que usaremos para el campo de Kalb-Ramond.}

\justify{Este estudio se propone elaborar y analizar lagrangianos de ciertos campos bosónicos. Por eso, los siguientes capítulos se proponen estudiar tres campos tipo en sus casos masivos y no masivos: los campos de spin-1 (Caps. \ref{Spin-1Massive} y \ref{Spin-1Massless}), de spin-2 (Caps. \ref{Spin-2Massive} y \ref{Spin-2massless}) y de Kalb-Ramond (Caps. \ref{KRMassive} y \ref{KRmassless}). En cada uno de estos capítulos se analiza el número de grados de libertad del campo, se expone la ecuación de movimiento de la teoría y se calcula el hamiltoniano. Además, para los casos no masivos, los capítulos finalizan con un estudio de las helicidades de las componentes, esta sección extra nos permite observar el comportamiento de spin en los campos de cada capítulo.}

\justify{Finalmente el trabajo termina con una conclusión final en el Cap. \ref{Conclusion} y la Bibliografía.}

\justify{\textbf{Notación y convenio:} A lo largo de esta tesis se trabaja con la velocidad de luz como unidad natural ($c = 1$). $D$ representa el número de dimensiones de la variedad, aunque se suele trabaja en cuatro dimensiones ($D = 4$). Se usa el convenio de tiempo positivo para la métrica $(+, -, -, -)$. Los índices griegos ($\alpha$, $\beta$, $\gamma$ ...) son espacio-temporales ($\alpha = 0, 1, \cdots D$), mientras que los índices latinos ($a$, $b$, $c$ ...) solo representan las componentes espaciales ($i = 1, \cdots, D-1$). El valor 0 de los índices representa la dirección temporal. Además, se usa la convención de simetrización: $(a, b) = ab + ba$, y de antisimetrización $[a, b] = ab - ba$.}

\newpage 