	
\pagestyle{empty}
\justify{D. Juan José Fernández Morales con DNI - garantiza al firmar este documento que en la realización del TFM que lleva por título \textit{"Teorías de Campos Bosónicos: Una introducción a los campos de Spin-1, Spin-2 y Kalb-Ramond."} se han respetado todos los derechos de otros autores a ser citados, cuando se han utilizado sus resultados o publicaciones.}

\vspace*{2cm}



\justify{\today \hspace{10cm} Fdo.:}

\newpage

% DUDA: Hay que hacerlo en español e inglés
\begin{center}
	\textbf{Abstract}
\end{center}

\justify{The present thesis aims to provide a comprehensive analysis of bosonic field theories, with a focus on the development of essential skills in the construction of such theories. Through the exploration of this topic, we seek to gain a deeper understanding of the current state of knowledge in the field and make significant contributions to our understanding of fundamental interactions.}

\justify{The structure of the thesis is divided into several chapters. In the first chapter, three key concepts for the analysis of the theories presented in the following chapters are introduced: the Ostrogradsky theorem, Wigner classification, and Hodge dual. And in the following chapters, the study of spin-1, spin-2, and Kalb-Ramond fields is considered both in their massive and massless forms.}

\justify{This work presents a methodology for the analysis of physical theories. The thesis focuses on the study of the degrees of freedom of the fields associated with the theories, examining both the physical and apparent degrees of freedom. The equation of motion is derived and the Hamiltonian of the theory is studied with the aim of verifying that it is free of Ostrogradsky instabilities. Additionally, the behavior of the spin of each individual component of the massless fields is analyzed, confirming that the fields exhibit the expected spin and identifying which components correspond to field degrees of freedom and the associated spin values.}

\justify{In the case of spin-1 and spin-2 fields, the behavior of the fields is explored from a general perspective for both the massive and massless cases. These fields are interesting to study for their relationship with particle interactions and their importance in understanding gravity and related theories. On the other hand, in the case of Kalb-Ramond fields, we focus on their relationships with the other fields already studied, as well as the connection between spin and the massive or massless nature of the Kalb-Ramond field.}

\justify{In summary, the thesis provides a comprehensive analysis of bosonic field theories, with a focus on developing essential skills in the construction of such theories. We aim to gain a deeper understanding of the current state of knowledge in the field and make significant contributions to our understanding of fundamental interactions.}

\justify{\textbf{Keywords: }Bosonic field theories, Degrees of Freedom, Spin-1 field, Spin-2 field, Kalb-Ramond field.}


\newpage

\begin{center}
	\textbf{Resumen}
\end{center}

\justify{La presente tesis tiene como objetivo proporcionar un análisis exhaustivo de las teorías de campo bosónico, con un enfoque en el desarrollo de habilidades esenciales en la construcción de dichas teorías. A través de la exploración de este tema, se busca obtener una comprensión más profunda del estado actual del conocimiento en el campo y hacer contribuciones significativas en nuestra comprensión de las interacciones fundamentales.}

\justify{La estructura de la tesis se divide en varios capítulos. En el primer capítulo, se introducen tres conceptos clave para el análisis de las teorías presentadas en los siguientes capítulos: el teorema de Ostrogradsky, la clasificación de Wigner y el dual de Hodge. En los capítulos siguientes, se considera el estudio de campos de spin-1, spin-2 y Kalb-Ramond tanto en su forma masiva como sin masa.}

\justify{En este trabajo se presenta una metodología para el análisis de teorías físicas. La tesis se centra en el estudio de los grados de libertad de los campos asociados a las teorías, examinando tanto los grados de libertad físicos como los aparentes. Se deriva la ecuación de movimiento y se estudia el hamiltoniano de la teoría con el objetivo de comprobar que este se encuentra libre de inestabilidades de Ostrogradsky. Además, se analiza el comportamiento del spin de cada componente individual de los campos no masivos, confirmando que los campos exhiben el spin esperado y identificando qué componentes corresponden a grados de libertad del campo y los valores de spin asociados.}

\justify{En el caso de los campos de spin-1 y spin-2, se explora el comportamiento de los campos desde una perspectiva general tanto para el caso masivo como sin masa. Estos campos son interesantes de estudiar por su relación con las interacciones entre partículas y su importancia en la comprensión de la gravedad y teorías relacionadas. Por otro lado, en el caso de los campos de Kalb-Ramond, nos centramos en sus relaciones con los otros campos ya estudiados, así como la conexión entre el spin y la naturaleza masiva o sin masa del campo de Kalb-Ramond.}

\justify{En resumen, la tesis proporciona un análisis exhaustivo de las teorías de campo bosónico, con un enfoque en el desarrollo de habilidades esenciales en la construcción de tales teorías. Se busca obtener una comprensión más profunda del estado actual del conocimiento en el campo y hacer contribuciones significativas en nuestra comprensión de las interacciones fundamentales.}

\justify{\textbf{Keywords: }Teorías de campo bosónico, Grados de Libertad, Campo de spin-1, Campo de spin-2 y Campo de Kalb-Ramond.}


\newpage
% AGRADECIMIENTOS
\justify{}