\chapter{Conclusión} \label{Conclusion}

\justify{En este trabajo de fin de máster, se ha llevado a cabo un estudio exhaustivo de las teorías de campo bosónico. Se ha desarrollado una metodología efectiva para su análisis y construcción, mediante la exploración de los conceptos clave y el examen de los grados de libertad de los campos bosónicos. Como resultado, se ha obtenido una comprensión más profunda de estas teorías, lo que permitirá avances significativos en la comprensión de las interacciones fundamentales. Este trabajo cumplió con su objetivo de proporcionar un análisis y brindar habilidades esenciales en la construcción de teorías de campo bosónico.}
%\justify{Tras el capítulo del campo Kalb-Ramond no masivo concluimos este trabajo de Fin de Master. Este trabajo se ha centrado en el estudio de las teorías de campo bosónico, y ha logrado proporcionar una metodología para su análisis y construcción. A través de la exploración de los conceptos clave y el análisis de los grados de libertad de los campos, se ha obtenido una comprensión más profunda de las teorías que permitirán tener contribuciones significativas en la comprensión de las interacciones fundamentales. La tesis ha logrado cumplir con su objetivo de proporcionar un análisis exhaustivo y desarrollar habilidades esenciales en la construcción de teorías de campo bosónico.}

%#######################################################################################################
\vspace{5mm}
\section{Recapitulación de los objetivos y alcances de la tesis.}
% I. Recapitulación de los objetivos y alcances de la tesis: resumen de los objetivos de la tesis, el alcance y los resultados obtenidos.
%#######################################################################################################

\justify{La tesis tiene como objetivo principal proporcionar una evaluación exhaustiva de las teorías de campo bosónico, con un énfasis en el desarrollo de habilidades esenciales para construir estas teorías. La estructura de la tesis está organizada en varios capítulos, cada uno de los cuales se centra en un aspecto específico del estudio.}
%\justify{El objetivo principal de la tesis fue proporcionar un análisis exhaustivo de las teorías de campo bosónico, con un enfoque en el desarrollo de habilidades esenciales en la construcción de dichas teorías. La estructura de la tesis se dividió en varios capítulos, cada uno de los cuales abordaba un aspecto específico del estudio.}

\justify{El primer capítulo introduce tres conceptos clave para el análisis de las teorías en los capítulos posteriores: el teorema de Ostrogradsky, la clasificación de Wigner y el dual de Hodge. Los siguientes capítulos examinan el estudio de los campos de spin-1, spin-2 y Kalb-Ramond tanto con masa como sin masa. Por lo tanto, esta tesis aborda un total de seis modelos matemáticos de teorías de campos con spin.}
%\justify{El primer capítulo introdujo tres conceptos clave para el análisis de las teorías presentadas en los siguientes capítulos: el teorema de Ostrogradsky, la clasificación de Wigner y el dual de Hodge. Los capítulos siguientes consideraron el estudio de campos de spin-1, spin-2 y Kalb-Ramond tanto en su forma masiva como sin masa.}

\justify{La metodología para el análisis de teorías físicas se presenta en los siguientes capítulos, enfocándose en el estudio de los grados de libertad de los campos asociados con las teorías. La ecuación de movimiento se deriva y se examina el hamiltoniano de la teoría para garantizar que esté libre de inestabilidades de Ostrogradsky. Además, se analiza el comportamiento del spin de cada componente individual de los campos sin masa, confirmando que los campos exhiben el spin esperado y identificando cuáles son los componentes correspondientes a los grados de libertad del campo y los valores de spin asociados.}
%\justify{En los capítulos siguientes se presentó una metodología para el análisis de teorías físicas, centrándose en el estudio de los grados de libertad de los campos asociados a las teorías. Se derivó la ecuación de movimiento y se estudió el hamiltoniano de la teoría con el objetivo de comprobar que este se encontraba libre de inestabilidades de Ostrogradsky. Además, se analizó el comportamiento del spin de cada componente individual de los campos no masivos, confirmando que los campos exhiben el spin esperado y identificando qué componentes corresponden a grados de libertad del campo y los valores de spin asociados.}

\justify{La tesis demuestra haber cumplido con sus objetivos y alcances previstos. Se realizó una evaluación exhaustiva de las teorías de campos bosónicos, y los resultados aportan un conocimiento valioso y significativo para tener un mejor entendimiento del marco teórico actual de los campos con spin, así como de una base para todos los posibles marco de investigación que actualmente hacen uso. En general, la tesis contribuye a ampliar nuestra comprensión, y nos permite cimentar las bases para un el futuro ámbito de la investigación.}
%\justify{En resumen, los objetivos y alcances de la tesis fueron cumplidos con éxito. La tesis proporciona una evaluación exhaustiva de las teorías de campo bosónico y hace contribuciones significativas en nuestra comprensión del marco teórico de los campos con spin.}
%\justify{En conclusión, los objetivos y alcances de la tesis fueron cumplidos de manera efectiva. La tesis proporcionó un análisis de las teorías de campo bosónico y hizo contribuciones significativas en la comprensión en el marco de la teoría de campos con spin.}

%#######################################################################################################
\vspace{5mm}
\section{Discusión de los resultados.}
% II. Discusión de los resultados: análisis detallado de los resultados obtenidos y su importancia en el contexto de la tesis.
%#######################################################################################################

\justify{Los resultados obtenidos en la tesis demuestran la eficacia de la metodología propuesta para el análisis. En cada campo, se ha realizado un análisis de los grados de libertad, cumpliendo con la clasificación de Wigner. Se ha derivado la ecuación de movimiento y se ha estudiado con éxito el hamiltoniano de la teoría, verificando que está libre de inestabilidades de Ostrogradsky en todos los campos. Además, se ha analizado el comportamiento del spin de cada componente individual en los campos no masivos, confirmando el spin esperado y identificando los componentes que corresponden a los grados de libertad del campo y sus valores de helicidad.}
%\justify{En términos de los resultados obtenidos, la tesis ha demostrado la efectividad de la metodología propuesta para el análisis. En todos los campos se han realizado un análisis de los grados de libertad, dejando a todos ellos de acuerdo a lo esperado por la clasificación de Wigner. Se ha derivado la ecuación de movimiento y se ha estudiado el hamiltoniano de la teoría con éxito, comprobando que este se encuentra libre de inestabilidades de Ostrogradsky de todos los campos. Además, se ha analizado el comportamiento del spin de cada componente individual de los campos no masivos, confirmando que los campos exhiben el spin esperado e identificando qué componentes corresponden a grados de libertad del campo y los valores de helicidad asociados.}

\justify{En los resultados, se pueden comparar las discrepancias y similitudes entre los casos masivos y no masivos de una teoría, destacando el campo de Kalb-Ramond, en el que se ha analizado cómo la teoría presenta un salto de spin entre el caso masivo y el no masivo. Se ha estudiado la relación entre la estructura tensorial del campo y su spin y grados de libertad según la clasificación de Wigner.}
%\justify{En los resultado obtenidos hemos podido comprobar las discrepancias e igualdades entre los casos masivos y no masivos de una teoría, destacando el campo de Kalb-Ramond, en el que hemos analizado como efectivamente la teoría presenta un salto de spin entre el caso masivo y el caso no masivo. Hemos estudiado la relación que existe entre la estructura tensorial propuesta para el campo con el spin que puede presentar, y con el número de grados de libertad exigidos por la clasificación de Wigner.}

\justify{La investigación sobre las restricciones de los grados de libertad aparentes del campo ha permitido observar diferencias entre los campos masivos y no masivos. Por ejemplo, el término de masa en el lagrangiano permite obtener las condiciones necesarias, pero rompe con la simetría \textit{gauge} del sistema, mientras que en el caso no masivo, la simetría \textit{gauge} está presente, pero no es suficiente para determinar todos los grados de libertad no físicos del sistema. Además, se ha observado en los tres campos no masivos cómo una buena elección del \textit{gauge} permite no solo restringir los grados de libertad aparentes, sino también obtener una ecuación de movimiento de tipo onda relativista que garantice el comportamiento local y relativista requerido y cumplido en cada modelo.}
%\justify{La búsqueda de las restricciones de los grados de libertad aparentes del campo nos ha permitido observar nuevamente diferencias entre los campos masivos y no masivos, como el término de masa del lagrangiano permite obtener las condiciones necesarias, pero rompe con la posible simetría \textit{gauge} del sistema, mientras que el caso no masivo presenta la simetría \textit{gauge} pero no es suficiente para determinar todos los grados de libertad no físico del sistema. Además, hemos podido observar en los tres campos no masivos como una buena elección del \textit{gauge} nos permite no solo restringir los grados de libertad aparentes, si no de obtener una ecuación de movimiento de tipo onda relativista que garantice el comporatmiento local y relativista exigido y cumplido en cada uno de los modelos.}

\justify{En este trabajo, se investigó el lagrangiano más general para los campos de spin-1 y spin-2 masivos. Se comprobó que las familias de lagrangianos convergían a uno único para cada campo, después de cumplir la condición de ser libre de "ghosts". Por lo tanto, se puede afirmar que las acciones de Proca y Fierz-Pauli masivo son los únicos lagrangianos libres de fantasmas para cada uno de los campos correspondientes. Esto también se aplica a los campos de Maxwell y Fierz-Pauli no masivo, los cuales se basaron en los resultados obtenidos de sus contrapartes masivas.}
%\justify{En los casos de spin-1 y spin-2 masivos hemos partido del lagrangiano más genérico que se podría describir, para luego comprobar que las posibles familias de lagrangianos convergían a uno de cada campo tras exigirles que fueran libre de \textit{ghosts}. Por tanto, podemos afirmar que la acción de Proca y la acción de Fierz-Pauli masivo no solo acciones para campos de spin-1 y spin-2 respectivamente, son los únicos lagrangianos libres de fantasmas para cada una de los campos propuestos. Esto ocurre de igual forma con el campo de Maxwell y el campo de Fierz-Pauli no masivo, los cuales se apoyaron respectivamente de los resultados obtenidos de sus casos masivos.}

\justify{Después de identificar los grados de libertad, se estudió el hamiltoniano de cada sistema y se verificó que todos tienen un límite inferior en su dominio, lo que indica que están libres de inestabilidades de Ostrogradsky y tienen un dominio positivo, clasificándose dentro de los dos grupos de la clasificación de Wigner.}
%\justify{Después de determinar los grados de libertad se estudió el hamiltoniano de cada uno de los sistemas, y se comprobó efectivamente que todos ellos se encuentran descritos con un límite inferior en su dominio, lo que nos permite afirmar que todos y cada uno de los modelos presentados en este trabajo se encuentran libre de inestabilidades de Ostrogradsky, si no que además su dominio es positivo, por lo que se encuentran dentro de los dos posibles grupos de la clasificación de Wigner presentados en este trabajo.}

\justify{Para los casos no masivos de cada uno de los capítulos, se obtuvo un resultado final, que consistió en el estudio de la helicidad de cada una de las componentes de cada campo. Este análisis nos permitió aprovechar los resultados previos obtenidos en cada capítulo, ofreciendo una síntesis y un buen uso de los resultados de cada modelo no masivo. Asimismo, se verificó efectivamente cómo el análisis de los grados de libertad coincide con los resultados obtenidos en la sección de helicidad de cada capítulo.}
%\justify{Para los casos no masivos de cada uno de los capítulos tuvimos un último resultado, el estudio de la helicidad de cada una de las componentes de cada campo. Este estudio nos permitió hacer uso de los resultados obtenidos previamente en cada capítulo, mostrando una recopilación y buen uso de los resultados de cada modelo no masivo. Además, efectivamente se comprobó como el análisis de los grados de libertad coincidían con los resultados obtenidos en la sección de helicidad de cada capítulo.}

\justify{Por último, el capítulo de Kalb-Ramond masivo presentó un resultado adicional, y se demostró que este lagrangiano es dual a la acción de Proca. Esto aclaró cualquier duda sobre la naturaleza de su spin, y proporcionó un primer resultado sobre las dualidades entre teorías.}
%\justify{Finalmente, el capítulo de Kalb-Ramond masivo presenta un resultado extra, y fuimos capaces de mostrar que este lagrangiano es dual al la acción de Proca. Despejando toda duda de la naturaleza de su spin, y dejando un primer resultado sobre dualidades entre teorías.}

\justify{En conclusión, todos los resultados fueron los esperados en un primer momento. Esta capacidad de obtener resultados en diversos campos fortalece nuestra comprensión del marco de campos de spin, y nos brinda una base sólida de conocimiento para diferentes ámbitos de la física teórica, física de partículas y física de altas energías.}
%\justify{En resumen, todos los resultados fueron los esperados en un primer momento. Esta capacidad de generar resultados a lo largos de distintos campos nos permite afianzar la comprensión que se tiene sobre el marco de campos de spin y nos otorga una base del conocimiento para diversos ámbitos de la física teórica, de la física de partículas y de la física de altas energías.}

%#######################################################################################################
\vspace{4mm}
\section{Posibles continuaciones y conclusión final.}
%#######################################################################################################

\justify{Este trabajo contribuye con una amplia comprensión fundamental para futuros avances en el ámbito de la investigación. La teoría de campos bosónicos es un marco teórico diverso y extenso que abarca gran parte de la física fundamental actual. Esta teoría es una herramienta efectiva para describir comportamientos relativistas y locales, como las interacciones fundamentales.}
%\justify{Este trabajo aporta un basto conocimiento, fundamental para cualquier continuación en el ámbito de la investigación. La teoría de campos bosónicos es un marco teórico muy rico y que abarca actualmente casi cualquier rama de la física fundamental. se trata de una herramienta excelente para poder describir comportamientos relativistas y locales, como por ejemplo ocurre con las interacciones fundamentales.}

% DUDA: CITAR LOS POSIBLES CAMPOS DE INVESTIGACION QUE ESTOY COMENTANDO
\justify{Este trabajo permite establecer dos posibles direcciones de investigación. Por un lado, podemos continuar el estudio de los campos bosónicos, incluyendo la generalización y análisis de los campos de spin mayor a dos o con diferentes estructuras tensoriales. Por otro lado, podemos profundizar en alguno de los campos establecidos, dada la correlación establecida en el trabajo entre la física teórica y los distintos campos investigados, como el campo de spin-1 relacionado con las interacciones del modelo estándar, el campo de spin-2 con la gravedad y el campo de Kalb-Ramond con las teorías de cuerdas.}
%\justify{En este trabajo nos permitiría establecer dos posibles continuaciones, por un lado, podríamos continuar con el estudio de los campos bosónicos. Existe toda una rama de la investigación en la generalización y análisis de los campos de spin mayor de orden dos, o con distintos y diversas estructuras tensoriales. O Por ejemplo, realizar la misma investigación con una generalización de las dimensiones estudiadas, con diferentes tipos de variedades y demás.}

%\justify{Por el otro lado, una vez establecido esta base podríamos recurrir a una mayor en alguno de los campos planteados. A lo largo del trabajo se ha establecido correlaciones entre la física teórica y los distintos campos que se han investigados. Como el campo de spin-1 se encuentra relacionado con las interacciones del modelo estándar, el campo de spin-2 con la gravedad y el campo de Kalb-Ramond con las teorías de cuerdas. El haber obtenido conocimientos sobre la base que conforman estas distintas ramas nos permitiría escoger y especializarse en cualquiera de los ámbitos.}

\justify{Este trabajo ha sido un camino de exploración y comprensión de las teorías bosónicas, una introducción satisfactoria a la teoría de campos de spin y un importante avance en el conocimiento del autor. Gracias a este trabajo de fin de máster, se ha podido aplicar los conocimientos adquiridos a lo largo de la carrera y el máster para avanzar en la trayectoria investigadora. Este trabajo nos ha permitido responder preguntas y tener una mejor comprensión de la física actual, y se presenta como una fuente de inspiración para futuras investigaciones más profundas y complejas. En resumen, este trabajo es un paso importante en el camino hacia un conocimiento más completo y profundo de la física.}
%\justify{Este trabajo ha sido un camino de exploración y compresión de las teorías bosónicas, una introducción satisfactoria teoría de campos de spin y un avance en el conocimiento del autor. Gracias a este trabajo de fin de máster se ha podido hacer uso de los conocimientos recabados a lo largo de la carrera así como en el máster, y avanzar en la trayectoria investigadora. Este tesis nos ha permitido responder preguntas y tener una mejor visión de la física actual, así como una fuente de curiosidad ante nuevas y más profundas preguntas.}

\begin{comment}
I. Recapitulación de los objetivos y alcances de la tesis: resumen de los objetivos de la tesis, el alcance y los resultados obtenidos.

II. Discusión de los resultados: análisis detallado de los resultados obtenidos y su importancia en el contexto de la tesis.

III. Interpretación de los resultados: interpretación de los resultados en términos de las teorías de campo bosónico, y su relación con los conceptos clave (teorema de Ostrogradsky, clasificación de Wigner, dual de Hodge).

IV. Contribuciones significativas: discusión de las contribuciones significativas a nuestra comprensión de las interacciones fundamentales y del estado actual del conocimiento en el campo.

V. Limitaciones de la tesis: discusión de las limitaciones de la tesis, incluyendo las limitaciones metodológicas, los errores potenciales y las áreas de investigación futura.

VI. Implicaciones prácticas: discusión de las implicaciones prácticas de los resultados obtenidos, incluyendo su potencial impacto en la física teórica y aplicaciones en otras áreas.

VII. Conclusiones finales: conclusiones finales y recomendaciones para futuras investigaciones.
\end{comment}

% Hablar de lo que hemos hecho en cada capítulo y de lo que hemos conseguido
	% En los capítulos de spin-1 y 2 hemos pasado del  lagrangiano más general a concluir con las únicas posibles soluciones
	% Hemos comprobado el uso de la clasificación de Wigner
	% La importancia de evitar las inestabilidades de Ostrogradsky y la importancia de tener el hamiltoniano acotado inferiormente.

% Hablar de las ventajas de poder introducirnos en las teorías de campos con spin

% Hablar de los posibles caminos que podemos tomar gracias al conocimiento adquirido

% Hablar de posibles continuaciones que se podrían hacer con este trabajo

%Hablar de que estos tres permiten tener toda la información de un tensor de orden 2 (hay algo en la tesis doctoral que copia al de bert)

