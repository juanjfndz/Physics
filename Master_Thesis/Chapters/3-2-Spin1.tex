%#####################################################################################################
\chapter{Campo de spin-1 sin masa.} \label{Spin-1Massless}
%#####################################################################################################

\justify{En este capítulo, se lleva a cabo un análisis del caso no masivo del campo de spin-1. Este tipo de teoría ha sido ampliamente estudiada debido a su relación con la realidad física, ya que los campos de spin-1 no masivos se encuentran en las teorías de campos bosónicos de dos de las fuerzas fundamentales de la física, el electromagnetismo y la fuerza nuclear fuerte. Las motivaciones físicas resultan suficientes para estudiar el caso sin masa del campo de spin-1.}

%\justify{En este capítulo se analiza el caso no masivo del campo de spin-1. Al igual que con el caso del capítulo anterior, este tipo de teoría se encuentra bien estudiada porque se trata de un modelo íntimamente relacionado con la realidad: los campo de spin-1 no masivos se encuentran en las teorías de los campos bosónicos de dos fuerzas fundamentales de la física, el electromagnetismo y la fuerza nuclear fuerte. Por tanto, existe motivaciones físicas suficientes para querer estudiar el caso sin masa del campo de spin-1.}

\justify{Al igual que en los capítulos anteriores relacionados con casos no masivos, este capítulo presenta un análisis de la teoría del campo de spin-1 no masivo, junto con una sección final en la que se examinan las helicidades de cada una de las componentes del campo de estudio. Para llevar a cabo este análisis, nos basaremos en los resultados obtenidos en el Capítulo \ref{Spin-1Massive}, que trata el caso de masa finita del campo de spin-1.}

%\justify{Como se describe en la introducción de esta tesis, los capítulos relacionados con casos no masivos presentan el mismo análisis de la teoría que el de los casos masivos, además de una última sección en la que analizar las helicidades de cada una de las componentes del campo de estudio. En este caso, al igual que en el resto de casos no masivos, nos apoyaremos de los resultados obtenidos de los casos de masa finita - en este caso del Cap. \ref{Spin-1Massive} - para el análisis de esta teoría.}


%#####################################################################################################
\vspace{4mm}
\section{Lagrangiano}
%#####################################################################################################

\justify{Estamos interesados en crear un modelo análogo al capítulo anterior, salvo que en esta teoría no se contempla el posible efecto masivo del campo. Es decir, buscamos modelar el comportamiento de un campo $A_{\mu}$ de spin-1 y sin masa, que viva en $\mathcal{M}_{4}$, y cuya acción sea local e invariante Lorentz.}

\justify{El lagrangiano más general que se puede describir bajo estas condiciones es de la forma}

\begin{equation} \label{eq:spin1massless:lagrangian0}
	\mathcal{L}^{\text{ spin}-1} = 
	-\frac{1}{2} \left( \partial_{\mu} A_{\nu} \, \partial^{\mu} A^{\nu} - \partial_{\mu} A^{\mu} \, \partial_{\nu} A^{\nu}\right),
\end{equation}

\justify{y resulta análogo al lagrangiano de Proca, expresado en la Ec. \eqref{eq:spin1mass:lagrangianoProca}, salvo por el término masivo. Debido a su enorme relevancia en la física de partículas, y de forma histórica con la interacción electromagnética, este lagrangiano también presentan nombre propio: se define como la acción de Maxwell, en nombre de James Clerk Maxwell.}

\justify{El primer prefactor se define bajo el paradigma de la normalización canónica, mientras que para la elección del segundo prefactor se toma en cuenta el análisis realizado en el Apto. \ref{apartado:spin1mass:ghostelimiantion}: la elección $\lambda_{2} = \frac{1}{2}$ previene las posibles inestabilidades de Ostrogradsky visualizadas en los términos dinámicos de este lagrangiano, al igual que también los previó en $\mathcal{L}^{\text{ spin}-1}_{\;m}$. }

\justify{Ante la idea de añadir posibles términos adicionales - por ejemplo $\partial_{\mu} A_{\nu} \partial^{\mu} A^{\nu}$ - podemos afirmar que se encuentran relacionados a los términos ya expuestos salvo un término de derivada total, como ocurre en la Ec. \eqref{eq:spin1:equivalencia} para el caso del capítulo anterior.}

\justify{Además, se puede definir el tensor de Faraday $F_{\mu\nu}$ como un tensor de orden dos, antisimétrico y que cumple la identidad de Bianchi}

\begin{equation} \label{eq:spin1massless:Bianchi}
	\varepsilon^{\lambda\rho\mu\nu} \partial_{\rho} F_{\mu\nu} = 0.
\end{equation}

\justify{El tensor de Faraday se expresa en función del campo $A_{\mu}$ y se define como}

\begin{equation}
	F_{\mu\nu} \equiv \partial_{[\mu}A_{\nu]} 
	\hspace{4mm} \Longrightarrow \hspace{4mm}
	\mathcal{L}^{\text{ spin}-1} =  -\frac{1}{4} F_{\mu\nu}F^{\mu\nu}.
\end{equation}

%#####################################################################################################
\vspace{4mm}
\section{Grados de Libertad.} \label{Spin1massless:DOF}
%#####################################################################################################

\justify{Una vez expuesto el lagrangiano, procedemos a analizar sus grados de libertad. Por un lado, el campo $A_{\mu}$ se comporta de forma análoga al campo del capítulo anterior: es un vector lorentziano con un máximo de cuatro grados de libertad, cuya álgebra de Lorentz asociada al campo de spin-1 en la representación irreducible de $\mathfrak{so}\,3$ es $\left(1\oplus0\right)$ y cuya representación de segundos números cuánticos viene dada por $\left\{(-1, 0, 1), (0)\right\}$.}

\justify{Por el otro lado, la clasificación de Wigner (ver Sección \ref{introduction:WignerClassification}) prevé dos grados de libertad para cualquier campo bosónico no masivo, independientemente de su spin. Además, como el campo $A_{\mu}$ tiene spin 1, las representaciones de los dos grados de libertad deben estar relacionadas con las helicidades $\pm 1$.}
%\justify{Por el otro lado, la clasificación de Wigner (Sec. \ref{introduction:WignerClassification})  predice 2 grados de libertad para cualquier campo bosónico no masivo, indistintamente del número de spin que presente. No solo eso, por ser $A_{\mu}$ un campo con spin-1 las representaciones de los dos grados de libertad deben relacionarse con las helicidades $\pm 1$.}

\justify{La diferencia en los grados de libertad entre el campo $A_{\mu}$ y los previstos por la clasificación de Wigner puede explicarse deduciendo que hay dos restricciones en la teoría. Según la imposición de Wigner, estas dos restricciones deben estar relacionadas con las dos representaciones de spin 0. Esta discrepancia en los grados de libertad nos lleva a profundizar en nuestro estudio del lagrangiano y a buscar sus condiciones de contorno para analizarla más a fondo.}
%\justify{La diferencia en los grados de libertad entre $A_{\mu}$  y los predichos por la clasificación de Wigner puede resolverse deduciendo que hay dos restricciones en la teoría. A partir de la imposición de Wigner, se deduce que estas dos restricciones deben estar relacionadas con las dos representaciones de spin-0. Esta discrepancia en los grados de libertad nos motiva a realizar nuestro estudio para analizar en mayor profundidad el lagrangiano así como buscar sus condiciones de contorno.}

\justify{En este contexto, es importante señalar que la primera diferencia que hemos encontrado entre el caso masivo y $A_{\mu}$ radica en el número de grados de libertad esperados por la clasificación de Wigner. En el caso masivo se espera un grado de libertad adicional al campo de esta sección. Esta desigualdad en los grados de libertad es un punto de inflexión clave entre ambas teorías. Además, es importante tener en cuenta que esta diferencia no desaparece a medida que se toma el límite $m \rightarrow 0^{+}$ del caso masivo. Es decir, la discrepancia se debe a que el planteamiento de una teoría masiva o no masiva ya condiciona los grados de libertad, independientemente del valor de la masa.}
%\justify{Esta también es la primera disonancia con el caso masivo, el número de grados de libertad esperado en $A_{\mu}$ resulta distinto al esperado en el campo del caso masivo del capítulo anterior. Esta desigualdad en los grados de libertad se trata de un punto de inflexión entre ambas teorías, hay que tener en cuenta que esta diferencia no desaparece tomando un límite $m \rightarrow 0$ del caso masivo, el haber planteado una teoría masiva ya condiciona los grados de libertad indistintamente del valor de la masa, aunque esta después sea nula. }

%#####################################################################################################
\vspace{4mm}
\subsection{Invarianza \textit{gauge.}}

% HABLAR DEL CASO MASIVO;, como este si que no presenta invarianza salvo el caso de añadir un campo auxiliar

% TEXTO: 

%\justify{, y esto presenta dos consecuencias: primero, con esta descomposición no se consigue eliminar un grado de libertad en el campo $A_{\mu}$, y segundo, el lagrangiano no resulta invariante frente a las transformaciones \textit{gauge} del tipo}

%\vspace{2mm}

%\begin{equation}
%	B_{\mu} \rightarrow B'_{\mu} + \partial_{\mu}\xi  
%	\hspace{5mm} \Longrightarrow \hspace{5mm}
%	\mathcal{L}^{\text{ spin}-1}_{\;m} (B'_{\mu} + \partial_{\mu}\xi  ) \neq  \mathcal{L}^{\text{ spin}-1}_{\;m} (B'_{\mu}), 
%\end{equation}

%\justify{porque el término de masa rompe esta posible simetría en el sistema. Como apunte, informar que hay extensiones de este tipo de teorías que sí permiten que campos masivos vectoriales presenten invarianza bajo este tipo de transformaciones, por ejemplo la acción de Stueckelberg \cite{Ruegg:2003ps}.

%#####################################################################################################

\justify{En este capítulo hemos obtenido la acción de Maxwell como la única opción posible que se encuentra libre de \textit{ghosts}. Para ello, se ha definido el prefactor de $\partial_{\mu}A^{\mu}\partial_{\nu}A^{\nu}$ en el lagrangiano. Esta elección se ha definido de tal manera que no aparezcan términos con derivadas temporales de orden dos cuando se realiza una descomposición del tipo}
%\justify{En este capítulo, hemos considerado la expresión del lagrangiano de nuestra teoría como la acción de Maxwell, dando un valor concreto al prefactor de $\partial_{\mu}A^{\mu}\partial_{\nu}A^{\nu}$ en el lagrangiano. Esta elección se basa en consideraciones técnicas para garantizar la coherencia de la teoría. Se ha definido de tal manera que no aparezcan términos con derivadas temporales de orden dos cuando se realiza una descomposición del tipo ecuación}
%\justify{El lagrangiano de nuestra teoría se expresa como la acción de Maxwell, esto implica que en este capítulo hemos tomado una predilección en el valor del prefactor de $\partial_{\mu}A^{\mu}\partial_{\nu}A^{\nu}$ del lagrangiano. Esta decisión se basa en consideraciones técnicas para garantizar la coherencia de la teoría, se encuentra definida de tal forma de que no aparezca términos con derivadas temporales de orden dos cuando se realiza una descomposición del tipo}

\begin{equation} \label{eq:spin1massless:Adecomposition}
	A_{\mu} = \mathcal{A}_{\mu}  + \partial_{\mu}\chi.
\end{equation}

\justify{Como ya se analizó en el capítulo anterior en las ecuaciones \eqref{eq:Spin1massless:chighost} y \eqref{eq:Spin1massless:chighost2}, el lagrangiano presentado no incluye términos dinámicos relacionados con el campo $\chi$ de la descomposición \eqref{eq:spin1massless:Adecomposition}. En otras palabras, el campo escalar desaparece trivialmente del lagrangiano $\mathcal{L}^{\text{ spin}-1}$ al elegir el prefactor $\lambda_{2} = \frac{1}{2}$. Y por tanto, podemos afirmar que el lagrangiano de Maxwell resulta invariante bajo transformaciones \textit{gauge} del campo $A_{\mu}$ del tipo}
%\justify{como se analizó en el capítulo anterior en las ecuaciones \eqref{eq:Spin1massless:chighost} y \eqref{eq:Spin1massless:chighost2}. Es más, este lagrangiano se encuentra descrito de tal forma que no aparece ningún término dinámico relacionado con el campo $\chi$ de la descomposición \eqref{eq:spin1massless:Adecomposition}. Es decir, el campo escalar desaparece trivialmente de $\mathcal{L}^{\text{ spin}-1}$ al elegir el prefactor $\lambda_{2} = \frac{1}{2}$. O lo que es lo mismo, el lagrangiano de Maxwell resulta invariante bajo transformaciones \textit{gauge} del campo $A_{\mu}$ del tipo}

\begin{equation} \label{eq:spin1massless:InvarianzaGauge}
	A_{\mu} \rightarrow A'_{\mu} = A_{\mu} + \partial_{\mu}\xi 
	\hspace{4mm} \Longrightarrow \hspace{4mm}
	\mathcal{L}^{\text{ spin}-1}  \left( A_{\mu} \right) = \mathcal{L}^{\text{ spin}-1}  \left( A'_{\mu} \right),
\end{equation}

\justify{donde $\xi$ es un parámetro escalar libre. Por tanto, La invariancia del lagrangiano de Maxwell ante transformaciones del tipo descritas en la ecuación \eqref{eq:spin1massless:InvarianzaGauge} permite redefinir el campo de la teoría de forma arbitraria sin afectar el sistema físico resultante.}
%\justify{El campo de fuerza $F_{\mu\nu}$ también resulta invariante bajo este tipo de transformaciones.}

\justify{La invariancia bajo transformaciones \textit{gauge} es una propiedad que no se presenta en el lagrangiano de Proca, como se puede ver en la ecuación \eqref{eq:spin-1massive:lambdacondition}. Esto se debe a que la invariancia de $\mathcal{L}^{\text{spin}-1}_{\;m}$ no se cumple en el término de masa. No obstante, es posible encontrar extensiones de la teoría de spin-1 masiva que permiten que los campos masivos vectoriales tengan invariancia bajo transformaciones gauge mediante el uso de un campo auxiliar, como ocurre en la acción de Stueckelberg \cite{Ruegg:2003ps}.}
%\justify{La invariancia bajo transformaciones gauge no es una característica presente en el lagrangiano de Proca, como pudimos observar en la ecuación \eqref{eq:spin-1massive:lambdacondition}. Esto se debe a que la invariancia de $\mathcal{L}^{\text{spin}-1}_{;m}$ se rompe en el término de masa. Sin embargo, es posible encontrar extensiones de una teoría de spin-1 masiva que sí permiten que campos masivos vectoriales presenten invarianza bajo transformaciones gauge gracias al uso de un campo auxiliar, como sí ocurre en la acción de Stueckelberg \cite{Ruegg:2003ps}.}
%\justify{ Esta invarianza no ocurre en el  lagrangiano de Proca, Ec. \eqref{eq:spin-1massive:lambdacondition}, porque la invarianza de $\mathcal{L}^{\text{ spin}-1}_{\;m}$ se rompe en el término de masa. Como apunte, informar que hay extensiones de este tipo de teorías que sí permiten que campos masivos vectoriales presenten invarianza bajo este tipo de transformaciones gracias a un campo auxiliar, por ejemplo la acción de Stueckelberg \cite{Ruegg:2003ps}.}



\justify{ Esta elección \textit{ad hoc} del campo nos otorga la posibilidad de restringir un grado de libertad al campo del sistema. El campo $\chi$ no aparece en el lagrangiano de Maxwell. Por tanto, su grado de libertad no se encuentra reflejado en la dinámica del sistema. Este resultado nos permite restringirlo, si substituimos la descomposición \eqref{eq:spin1massless:Adecomposition} sobre el campo $A'_{\mu}$ obtenido en la transformación \textit{gauge} \eqref{eq:spin1massless:InvarianzaGauge}, de esta forma conseguimos describir al nuevo campo como}
%\justify{Una forma de comprender esta restricción: el campo $\chi$, que presenta un grado de los cuatro de $A_{\mu}$, no aparece en el lagrangiano de Maxwell; por lo que puede presentar el valor que desee que no se verá reflejado en la dinámica del sistema. Si se estudia la descomposición del tipo ecuación \eqref{eq:spin1massless:Adecomposition} del campo resultante de una transformación del tipo ecuación \eqref{eq:spin1massless:InvarianzaGauge} se obtiene}

\begin{equation}
	 A'_{\mu} 
	= \mathcal{A}_{\mu}  + \partial_{\mu} \left( \chi + \xi \right).
\end{equation}

\justify{En la descomposición de este nuevo campo se puede observar que la nueva componente escalar $\left( \chi' \equiv \chi + \xi \right)$ sí depende de nuestra elección arbitraria. La componente escalar del campo sigue sin aparecer en la dinámica del sistema - es por ello que podemos realizar estas transformaciones - pero su valor, ahora sí, se encuentra restringido a la elección del autor, por lo que ya no es un grado de libertad de $A'_{\mu}$.}

\justify{En el contexto de la teoría del campo de spin-1 no masivo, no existe experiencia empírica que permita medir directamente el campo $A_{\mu}$. Esto significa que cualquiera de los infinitos campos $A'_{\mu}$ resultantes de una transformación \textit{gauge} son igualmente válidos para expresar dicha teoría. Además, la presencia de una restricción en el campo $A_{\mu}$ que depende de una elección \textit{ad hoc} sugiere que este campo no es una entidad física real, sino más bien un artefacto matemático o un campo auxiliar que nos permite estudiar la realidad física de manera más efectiva.}
%\justify{En la realidad no existe experiencia empírica con el campo $A_{\mu}$ que permita medirlo, por lo que cualquiera de los infinitos campos $A'_{\mu}$ resultan igual de válidos que $A_{\mu}$ para expresar la teoría. El hecho de que el campo $A_{\mu}$ presente una restricción que dependa de una elección \textit{ad hoc} implica que el campo fundamental de la teoría no es un campo físico, se trata de un artificio matemático un campo auxiliar que permite el estudio de la realidad física. }

%#####################################################################################################
\vspace{4mm}
\subsection{Ecuación de movimiento y Elección del \textit{gauge}.} \label{Spin1massless:DOF:EOM}
%#####################################################################################################

\justify{En esta sección se examina la ecuación de movimiento correspondiente a un campo de spin-1 no masivo. Para diversificar el análisis y evitar una repetición del cálculo relacionado al caso masivo, se explorará una situación distinta al caso libre que plantemos en el capítulo anterior \ref{Spin-1massive:EOMSubsection}. Para esta ocasión, tomaremos el lagrangiano general de spin-1 acoplado a una corriente y observaremos como deducimos las condiciones del lagrangiano de Maxwell. Este análisis permitirá introducir el teorema de Noether, ya que nos permitirá observar como la conservación de la corriente tiene una relación directa con las simetrías observadas anteriormente}
%\justify{Para evitar repetir el cálculo con respecto al caso masivo, exploraremos un sistema distinto y analizaremos el caso del campo de spin-1 acoplado a una corriente. De esta manera, podremos tener un primer contacto con el teorema de Noether y observar cómo efectivamente la conservación de la corriente se encuentra vinculada a una simetría del sistema.}

\justify{Para llevar a cabo este análisis, se empleará un lagrangiano más general que el de Maxwell, que incluye el término $\lambda_{2}$ aún sin definir, y una interacción del campo $A_{\mu}$ con una corriente $j_{\mu}$, como se ilustra a continaución}

\begin{equation}
	\mathcal{L}^{\text{ spin}-1} = 
	-\frac{1}{2} \partial_{\mu} A_{\nu} \, \partial^{\mu} A^{\nu} + \lambda_{2} \, \partial_{\mu} A^{\mu} \, \partial_{\nu} A^{\nu} - A_{\mu}j^{\mu}.
\end{equation}

\justify{La cinemática de este sistema se obtiene a partir de las ecuaciones de Euler-Lagrange y da lugar a }

\begin{equation}
	- j^{\alpha} - 2 \lambda_{2} \partial_{\beta}\partial^{\alpha}A^{\beta} + \partial_{\beta}\partial^{\beta}A^{\alpha} = 0.
\end{equation}

\justify{Si asumimos conservación de la corriente ($\partial_{\beta}j^{\beta} = 0$), podemos extraer del resultado la condición sobre el prefractor $\lambda_{2}$}

\begin{equation}  \label{eq:spin1massless:Lagrangiancondition} 
	 ( 2 \lambda_{2} -  1)\, \partial_{\alpha}\partial^{\alpha}\partial_{\beta}A^{\beta}   
	 = \partial_{\beta}j^{\beta} 
	 =  0 
	 \hspace{2mm} \xrightarrow{\forall A^{\beta}} \hspace{2mm}  \lambda_2 =  \frac{1}{2},
\end{equation}

\justify{que coincide con el valor predispuesto en la Ec. \eqref{eq:spin1massless:lagrangian0}. De esta forma, la ecuación de movimiento del campo $A_{\mu}$ se expresa como}

\begin{equation} \label{eq:spin1:solEomSpin1massless}
	 \partial_{\beta}\partial^{\beta}A^{\alpha}  - \partial_{\beta}\partial^{\alpha}A^{\beta} \equiv \partial_{\beta}F^{\beta\alpha} = j^{\alpha}.
\end{equation}

\justify{A partir de estos resultados, es evidente que la conservación de la corriente $j^{a}$ implica una restricción específica sobre el parámetro $\lambda_{2}$, lo que lleva a la conclusión de que el lagrangiano de Maxwell es el único candidato viable dentro del espectro de posibilidades planteadas al inicio de esta subsección. En este sentido, se manifiesta en el sistema una correspondencia entre la conservación de la corriente $j^{a}$ y la simetría \ref{eq:spin1massless:InvarianzaGauge} presente en el lagrangiano, este resultado se encuentra en consonancia con las implicaciones del teorema de Noether.}
\justify{Con estos resultados, apreciamos que la conservación de la corriente $j^{a}$ implica la imposición de una condición sobre el parámetro $\lambda_{2}$, lo que deja al lagrangiano de Maxwell como único candidato del abaníco de posibildades ofrecidos al comienzo de esta subseción. Por tanto, podemos observar en este sistema una relación entre la conservación de la corriente $j^{a}$ y la simetría debida a la invarianza \ref{eq:spin1massless:InvarianzaGauge} del lagrangiano, como expresa el teorema de Noether.}
%\justify{Con estos resultados, apreciamos que la conservación de la corriente $j^{a}$ implica la imposición de una condición sobre el parámetro $\lambda_{2}$. Sin embargo, es precisamente este parámetro el que nos permite que el lagrangiano sea invariante bajo transformaciones gauge. Es decir, el lagrangiano presenta una simetría interna en el sistema, como ya nos informaba el teorema de Noether. En otras palabras, en este sistema la conservación de la corriente $j^{a}$ se relaciona con la simetría \textit{gauge} del lagrangiano, como se muestra en la ecuación \ref{eq:spin1massless:InvarianzaGauge}.}

\justify{A pesar de haber encontrado un ejemplo claro de las consecuencias del teorema de Noether, aún no hemos podido establecer ninguna condición que nos permita restringir el segundo grado del campo $A_\mu$. Para lograrlo, debemos imponer una condición adicional que cumpla con los siguientes requisitos: no debe cancelar la condición obtenida bajo la invarianza  \eqref{eq:spin1massless:InvarianzaGauge}, y debe estar relacionada con una componente del campo $A_\mu$ que se corresponda con la representación de spin-0. En la bibliografía, este proceso se conoce como elección o fijación del \textit{gauge}.}
%\justify{La falta de una condición extra que permita la eliminación de un segundo grado de libertad del campo $A_{\mu}$ permite el imponer bajo elección una segunda restricción. La imposición debe encontrarse de acuerdo a lo planteado hasta ahora en el análisis: no debe cancelar la condición obtenida bajo la invarianza global \eqref{eq:spin1massless:InvarianzaGauge}, y debe encontrarse relacionado con una componente del campo $A_{\mu}$ que se relacione con la representación de spin-0, en la bibliografía a esto se le refiere como escoger o fijar el \textit{gauge}.}

\justify{Existen muchas restricciones que permiten obtener dicha condición, como pueden ser el \textit{gauge} de Coulomb, el \textit{gauge} de Weyl o el conjunto de \textit{gauges} $R_{\xi}$. A pesar del abanico de posibilidades, en el comienzo de la teoría decidimos imponer que se tratase de un modelo relativista, de los \textit{gauges} posibles existe uno que resulta manifiestamente invariante Lorentz y que además presenta relación con el caso masivo: el \textit{gauge} de Lorenz }

\begin{equation} \label{eq:spin1massless:lorenzgauge}
	\partial_{\mu}A^{\mu} = 0.
\end{equation} 

\justify{A este \textit{gauge} se le denomina incompleto, esto se debe a que la elección de esta condición no permite fijar directamente el término $\xi$ de la transformación \textit{gauge} de la Ec. \eqref{eq:spin1massless:InvarianzaGauge}.}

\justify{La elección del gauge de Lorenz es análoga a la condición obtenida en el caso masivo en la ecuación \eqref{eq:spin1mass:condition}, de tal forma que podemos afirmar que esta condición es capaz de restringir un grado de libertad al campo $A_{\mu}$ (como observamos en el ejemplo \ref{eq:LorenzProof}). Además, esta condición nos permite simplificar la ecuación de movimiento \eqref{eq:spin1:solEomSpin1massless}, para el caso sin corriente $j^{\mu}$ la ecuación de movimiento se presenta como }
%\justify{La elección de este \textit{gauge} es análoga a la condición obtenida en el caso masivo en la Ec. \eqref{eq:spin1mass:condition}, de tal forma que podemos afirmar que esta condición es capaz de restringir un grado de libertad al campo $A_{\mu}$ (como observacmos en el ejemplo \eqref{eq:LorenzProof}), y además nos permite simplificar la ecuación de movimiento \eqref{eq:spin1:solEomSpin1massless} a}

\begin{equation} \label{eq:spin1:solEomSpin1massless2}
	\partial_{\mu}\partial^{\mu}A^{\nu} = 0,
\end{equation}

\justify{una solución de tipo relativista e invariante Lorentz.}

%#####################################################################################################
\vspace{4mm}
\subsection{Grados de libertad.} 
%#####################################################################################################

\justify{En esta sección, se ha demostrado que el campo de spin-1 sin masa presenta dos grados de libertad, lo que concuerda con la clasificación de Wigner para esta clase de partículas. La invariancia \textit{gauge} \eqref{eq:spin1massless:InvarianzaGauge} y la elección del \textit{gauge} \eqref{eq:spin1massless:lorenzgauge} de Lorenz restringen un grado de libertad cada uno. En la sección \ref{Spin1:Helicity}, se lleva a cabo un análisis detallado del campo que confirma esta afirmación y muestra la relación entre las componentes del campo y la helicidad de las mismas.}
%\justify{El análisis realizado en esta sección nos permite afirmar que el campo de spin-1 sin masa presenta dos grados de libertad, coincidiendo con el número esperado por la clasificación de Wigner para este tipo de partículas. Tanto la invarianza \textit{gauge} como la elección del \textit{gauge}de Lorenz permiten restringir un grado de libertad cada uno. En la sección \ref{Spin1:Helicity} se hace un estudio componente a componente del campo que lo reafirma y muestra la relación de las componentes del campo con la helicidad de estas.}

\justify{En este trabajo hemos podido examinar el parentesco entre los campos vectoriales de spin-1 en el caso no masivo y masivo. Se utilizó el caso masivo como guía para facilitar los cálculos realizados en el caso no masivo. Esto nos permitió evitar posibles inestabilidades en el lagrangiano al plantearlo, dando lugar al lagrangiano de Maxwell como la única solución posible para un campo vectorial sin masa y con spin-1. Además, el lagrangiano de Proca y su condición de Lorenz nos guió para establecer el \textit{gauge} adecuado para que la ecuación de movimiento cumpliera con las condiciones previamente establecidas en la teoría, como la necesidad de que la solución sea relativista.}
%\justify{Además, en esta sección se puede observar el parentesco entre el caso no masivo y masivo de spin-1. El caso masivo ha servido como guía para facilitar los cálculos realizados en el caso no masivo: por un lado, nos ha permitido evitar posibles inestabilidades en el lagrangiano nada más plantear el lagrangiano. Dando lugar al lagrangiano de Maxwell como única solución posible para un campo vectorial sin masa y con spin-1. Por otro lado, el lagrangiano de Proca ha ayudado a establecer el \textit{gauge} adecuado para que la ecuación de movimiento cumpliera con las condiciones previamente establecidas en la teoría como que la solcuión debe ser relativista.}
%\justify{De esta sección también podemos confirmar el parentesco entre el caso no masivo y masivo de spin-1. El caso masivo nos ha servido de guía para facilitar los cálculos obtenidos del caso no masivo, nos ha prevenido de las posibles inestabilidades a las que se tenía que enfrentar el lagrangiano que nos ha permitido obtener el lagrangiano de Maxwell como única solución para un campo vectorial sin masa y con spin-1. E incluso, el lagrangiano de Proca nos ha permitido saber como fijar el \textit{gauge} para que la ecuación de movimiento se ajustara a las condiciones que impusimos previamente a la teoría.}

\justify{Finalmente, llegamos a la conclusión de que los campos vectoriales de spin-1 solo pueden ser descritos por dos tipos de lagrangianos: el lagrangiano de Proca para el caso masivo y el lagrangiano de Maxwell para el caso no masivo.}
%\justify{De esta forma, llegamos a la conclusión de que los campos vectoriales de spin-1 solo pueden estar descritos por dos tipos de lagrangianos, el lagrangiano de Proca para el caso masivo y el lagrangiano de Maxwell para el caso no masivo.}

%#####################################################################################################
\vspace{4mm}
\section{Hamiltoniano.} \label{Spin1:H}
%#####################################################################################################

\justify{En estos capítulos se han mencionado los problemas de inestabilidad de Ostrogradsky, es decir, que el hamiltoniano del sistema no tenga una cota inferior bien definida. Resulta fundamental que las teorías clásicas se encuentre bien descritas, y una de estas condiciones es que el sistema tenga un estado fundamental de mínima energía.}
%\justify{En el actual paradigma de la física, importa que una teoría clásica se encuentre bien descrita, y entre estas condiciones se encuentra la idea de que el sistema presenta un estado fundamental, un estado de mínima energía. Tanto en el capítulo anterior como en este se han comentado las problemáticas de que la teoría presenten inestabilidades de Ostrogradsky. Es decir, que el hamiltoniano del sistema no presente una cota inferior bien definida.}

\justify{En nuestra teoría de campos de spin-1, es importante analizar el hamiltoniano para asegurar la conformidad con el grupo de clasificación de Wigner asociado a energía positiva. Es decir, es crucial verificar que existe una cota inferior del hamiltoniano y que éste se encuentra definido positivamente. Por lo tanto, se hace necesario calcular y analizar el hamiltoniano de $\mathcal{L}^{\text{ spin}-1}$.}
%\justify{Como la teoría que estamos construyendo debe encontrarse en de acuerdo con el grupo de la clasificaión de Wigner asociados a energía positiva, existe una motivación para calcular y analizar el hamiltoniano de $\mathcal{L}^{\text{ spin}-1}$. No dejaremos escapar la oportunidad de comprobar que efectivamente no solo existe una cota inferior del hamiltoniano, si no que este se encuentra definido positivo.}

\justify{El hamiltoniano se puede obtener de forma análoga al caso masivo en el desarrollo \eqref{eq:spin1mass2:Hamiltoniano}, en este caso el resultado es}

\begin{equation} \label{eq:spin1:Hamiltoniano}
	\mathcal{H^{\text{ spin-1}}} = \frac{1}{2} \left(F_{0i}\right)^{2} + \frac{1}{4}\left(F_{ij}\right)^{2} + \mathcal{H}_{\text{Boundary}}.
\end{equation}

\justify{Donde $\left(F_{0i}\right)^{2}$ y $\left(F_{ij}\right)^{2}$ son definidos completamente positivos por convenio. El cálculo del hamiltoniano del campo de spin-1 se ha llevado a cabo mediante la consideración de la ecuación de movimiento \eqref{eq:spin1:solEomSpin1massless2}. Como resultado, se ha obtenido un hamiltoniano definido positivamente, con excepción de un término de frontera que depende de las condiciones de contorno aplicadas. Este hallazgo se ha realizado en la capa de masas del sistema.}
%\justify{mostrando a un hamiltoniano definido positivamente salvo un término de frontera, este último término dependerá de las condiciones de contorno a imponer. Este resultado se encuentra en la capa de masas debido a que se ha tenido en cuenta la ecuación de movimientos \eqref{eq:Spin1massless:EOMF} para su desarrollo.}

\justify{El lagrangiano utilizado en este análisis parece ser una opción adecuada para modelizar el comportamiento del campo físico en cuestión, ya que presenta los grados de libertad apropiados y un hamiltoniano y ecuación de movimiento definidos. En el siguiente apartado, se analizará el comportamiento específico del campo de spin-1, examinando cómo se manifiestan sus comportamientos de spin.}
%\justify{En este punto del análisis, el lagrangiano parece ser un prometedor candidato para el campo físico que se quiere modelizar. Se trata de un lagrangiano que puede presentar los grados de libertad adecuados, con un hamiltoniano no problemático y con una ecuación de movimiento definida. En el siguiente apartado se analiza el comportamiento en sí del campo, observaremos como se manifiesta sus comportamientos de spin.}

%#####################################################################################################
\vspace{4mm}
\section{Helicidad.} \label{Spin1:Helicity}
%#####################################################################################################

\justify{En este capítulo, nuestro objetivo final es demostrar que el campo $A_{\mu}$ presenta un comportamiento de spin-1. Para lograrlo, utilizaremos los resultados obtenidos a lo largo del capítulo para aplicarlos de forma práctica a nuestro campo de estudio. El procedimiento que seguiremos es análogo al presentado en el capítulo 17 del libro \cite{JanssenBook}.}
%\justify{El último objetivo de este capítulo es comprobar que  $A_{\mu}$ presenta un comportamiento de spin-1. Para ello, se hace acopio de los resultados que se han obtenido a lo largo de este capítulo y se emplean de forma práctica sobre nuestro campo de estudio. El procedimiento que se plantea resulta análogo al presentado en el capítulo 17 de \cite{JanssenBook,}.}

%#####################################################################################################
\vspace{4mm}
\subsection{Primera restricción.}
%#####################################################################################################

\justify{En el Apto. \ref{Spin1massless:DOF:EOM} argumentamos que en este modelo se tomaría el \textit{gauge} de Lorenz \eqref{eq:spin1massless:lorenzgauge}, de esta forma la ecuación de movimiento resulta en una ecuación de ondas relativista cuya solución es una combinación lineal de ondas planas. Para simplificar, en este apartado asumiremos que la solución es de la forma}

\begin{equation} 
	A_{\mu} = C_{\mu} e^{ik_{\lambda}x^{\lambda}},
\end{equation}

\justify{donde las componentes de la amplitud $C_{\mu}$ son constantes y $k_{\mu}$ el vector de onda. La nueva expresión del campo permite visualizar mejor cómo el \textit{gauge} de Lorenz elimina un grado de libertad.}

\begin{itemize}

	\item Si se aplica la condición \textit{gauge} \eqref{eq:spin1massless:lorenzgauge} sobre el campo 

	\begin{equation} \label{eq:Spin1massless:Condition1}
		0 = \partial_{\mu}A^{\mu} = 
		C_{\mu} \partial_{\mu} \left(e^{ik_{\lambda}x^{\lambda}}\right) = 
		iC_{\mu} k^{\mu} e^{ik_{\lambda}x^{\lambda}} \iff C_{\mu} k^{\mu} = 0,
	\end{equation}
	
	observarmos que la amplitud y el vector de onda resultan perpendiculares entre ellos.

	\item Si se aplica la ecuación de movimiento \eqref{eq:spin1:solEomSpin1massless2} se obtiene que el vector de onda $k_{\mu}$ es nulo

	\begin{equation} \label{eq:Spin1massless:Condition2}
		0 = \partial_{\nu}\partial^{\nu}A^{\mu} = 
		C_{\mu} \partial_{\nu}\,\partial^{\nu} \left(e^{ik_{\lambda}x^{\lambda}}\right) = 
		-C_{\mu} k_{\nu}\,k^{\nu} e^{ik_{\lambda}x^{\lambda}} \iff k_{\nu}k^{\nu}= 0,
	\end{equation}

	y por tanto la velocidad de propagación del campo sobre el espacio-tiempo es a velocidad de la luz.

\end{itemize}

\justify{Aunque las ecuaciones \eqref{eq:Spin1massless:Condition1} y \eqref{eq:Spin1massless:Condition2} se traten de dos condiciones distintas en conjunto se comporta como una ligadura sobre una de las componentes de $A_{\mu}$. Con ambas condiciones se pueden fijar una componente de $k_{\mu}$ y de $C_{\mu}$ - por ejemplo, la componente temporal -  y dejarlas en función de las otras tres coordenadas. }


\justify{Si por ejemplo se tomara la cuarta componente de $k_{\mu}$ como la única dirección espacial arbitraria del vector de onda, la única configuración posible para que el vector de onda sea nulo es}

\begin{equation}
	(k^{\mu}) = \begin{pmatrix}
					k \\
					0 \\
					0 \\
					k 
			        \end{pmatrix},
\end{equation}

\justify{y en consecuencia, la Ec. \eqref{eq:Spin1massless:Condition1} sería de la forma}

\begin{equation}
	k \left( C_{0} + C_{3} \right) = 0 \hspace{5mm} \iff \hspace{5mm} C_{0} = -C_{3}.
\end{equation}

\justify{Estos dos últimos resultados dejan a la componente $A_{0}$ completamente atada a la componente $A_{3}$

\begin{equation} \label{eq:spin1massless:A-1}
	(A_{\mu}) = (-A_{3},\, A_{1},\, A_{2},\, A_{3}).
\end{equation}

\justify{En general, se puede dejar a una de las componentes del campo $A_{\mu}$ completamente determinada por el resto componentes, mostrando que el \textit{gauge} de Lorenz reduce al campo un grado de libertad. Para nuestro caso, el campo $A_{\mu}$ queda con los tres grados de libertad asociados a las componentes espaciales.}

%#####################################################################################################
\vspace{4mm}
\subsection{Segunda restricción.}
%#####################################################################################################

\justify{Como se mencionó en la Sección \ref{Spin1massless:DOF:EOM}, la transformación de Lorenz no determina completamente el gauge. Aún subsiste libre el término $\xi$ en la transformación gauge \eqref{eq:spin1massless:InvarianzaGauge}. Una elección adecuada de este parámetro permitiría ajustar el grado de libertad adicional al que pronosticamos con la clasificación de Wigner. Para lograrlo, empleamos la siguiente transformación en el campo $A_{\mu}$:}

\begin{equation}
	\left.
	\begin{array}{lr}
		A'_{\mu} &= A_{\mu} + \partial_{\mu}\xi\\
				      \\
		\xi &=  iU\, e^{ik_{\lambda} x^{\lambda}}
  	\end{array}
	\right\}
	\hspace{2mm} \Longrightarrow \hspace{2mm}
	A'_{\mu} = \left(C_{\mu} - k_{\mu}U\right)\, e^{ik_{\lambda}x^{\lambda}} \equiv C'_{\mu} \, e^{ik_{\lambda}x^{\lambda}},
\end{equation}

\justify{donde la amplitud $U$ es una constante indefinida. Resaltamos que esta particular elección de $\xi$ garantiza que el nuevo campo transformado aún cumpla con la condición \textit{gauge} de Lorenz}

\begin{equation}
	\partial_{\mu}A'^{\mu} = 0,
\end{equation}

\justify{En consecuencia, esta elección de $\xi$ permite que la transformación mantenga los resultados \eqref{eq:Spin1massless:Condition1} y \eqref{eq:Spin1massless:Condition2}.}

\justify{Si volvemos a considerar un desplazamiento del campo solamente en la última coordenada espacial, la amplitud de $A'_{\mu}$ adopta la forma}

\begin{equation}
	(C'_{\mu}) = \left( - (C_{3} + kU), \; C_{1},\;  C_{2},\;   C_{3} + kU \right),
\end{equation}

\justify{perdiendo nuevamente un grado de libertad, en este caso asociado a la componente temporal. Nuevamente, observamos que la selección de la dirección de $k_{\mu}$ del campo, en conjunto con los resultados de las ondas relativistas ( \eqref{eq:Spin1massless:Condition1} y \eqref{eq:Spin1massless:Condition2}), indica que la componente temporal se encuentra ligada a la última componente espacial. Esto sugiere que el campo no puede ser libremente configurado y que estas componentes se encuentran interrelacionadas. Esto concuerda con lo esperado por la Ec. \eqref{eq:spin1massless:A-1}.}

\justify{En comparación con la subsección anterior, este nuevo caso es notable debido al campo transformado. Esta nueva perspectiva altera la configuración de las componentes interdependientes de $A'_{\mu}$, ya que la última componente no depende solamente de la amplitud $C_{3}$, sino también del parámetro $U$ no definido. La tentación es clara, si relacionamos $U$ con la amplitud de la última componente espacial, nos permitiría la eliminación de dos componentes de $A'_{\mu}$}

\begin{equation}
	U = - k^{-1} C_{3}
	\hspace{5mm} \Longrightarrow \hspace{5mm}
	(C'_{\mu}) = \left( 0, \; C_{1},\;  C_{2},\;  0\right).
\end{equation}

\justify{Finalmente, obtenemos una representación visual que confirma que los grados de libertad del campo $A'_{\mu}$ son dos, tal como se discutió en la sección \ref{Spin1massless:DOF} y se pronosticó en la Sec. \ref{introduction:WignerClassification}. Gracias a este desarrollo, se hace evidente que las componentes del campo que persisten son aquellas perpendiculares al vector de onda, en nuestro caso específico: perpendiculares a la tercera componente espacial.}

%#####################################################################################################
\vspace{4mm}
\subsection{Helicidides del campo de spin-1.}}
%#####################################################################################################

\justify{Queda comprobar que efectivamente las componentes del campo $A'_{\mu}$ asociadas a los grados de libertad también son las que presentan las helicidades $\pm 1$. La visualización de estas helicidades viene dado por la aplicación de una rotación al campo bajo el grupo de transformaciones de Lorentz. Si la transformación de campo al rotar presenta la forma}
%\justify{Queda comprobar que efectivamente las componentes del campo $A'_{\mu}$ asociadas a los grados de libertad también son las que presentan las helicidades $\pm 1$. La visualización de estos spins viene dado por la aplicación de una rotación al campo bajo el grupo de transformaciones de Lorentz. Si la transformación de campo al rotar presenta la forma}

\begin{equation}
	A''_{\mu} = (\Lambda^{-1})^{\nu}_{\hspace{2mm}\mu} A'_{\nu} = e^{ih\theta} A'_{\mu},
\end{equation}

\justify{donde $(\Lambda^{-1})^{\nu}_{\hspace{2mm}\mu}$ representa la rotación del campo alrededor de la dirección de propagación, y $\theta$ se define como el ángulo de rotación de dicha rotación. Definiremos $h$ como la helicidad de cada una de las componentes del campo. En nuestro caso, el campo transforma bajo una rotación de un ángulo $\theta$ en el plano $x_{1}x_{2}$. El tensor de transformación se expresa}
%\justify{donde (\Lambda^{-1})^{\nu}_{\hspace{2mm}\mu} es la rotación, en este caso del campo, alrededor de la dirección de propagación y $\theta$ se define como el ángulo de rotación, definiremos $h$ como la helicidad de cada una de las componente del campo. En nuestro caso se trata de una rotación de un ángulo $\theta$ sobre el plano $x_{1}x_{2}$, el tensor de transformación se expresa como} 

\begin{equation} \label{eq:spin1massless:rotation}
	(\Lambda^{-1})^{\nu}_{\hspace{2mm}\mu} = \begin{pmatrix}
							1 & 0 & 0 & 0  \\
							0 & \cos{\left(\theta\right)} & -\sin{\left(\theta\right)} & 0 \\
							0 & \sin{\left(\theta\right)} & \cos{\left(\theta\right)} & 0\\
							0 & 0 & 0 & 1
		       			            \end{pmatrix} .
\end{equation}

\justify{Si se aplica la transformación en las componentes del campo que no presentan los grados de libertad podemos observar que resultan invariantes bajo esta transformación}

\begin{equation}
	C''_{0} = C'_{0}, \hspace{5mm}
	C''_{3} = C'_{3},
\end{equation}

\justify{es decir, que la helicidad de ambos es $0$, se trata de los spin-0 de las representaciones del triplete y el singlete. Las otras dos componentes del campo quedan al transformase de la siguiente forma}

\begin{align} \label{eq:spin1massless:secondtrans}
	C''_{1} &=  C'_{1}\cos{\left(\theta\right)} - C'_{2}\sin{\left(\theta\right)}, \\
	C''_{2} &=  C'_{1}\sin{\left(\theta\right)} + C'_{2}\cos{\left(\theta\right)}.
\end{align}


%%%%%%%%%%%%%%%%%%%

\justify{A primera vista, no se percibe claramente la helicidad de cada una de las componentes. Sin embargo, es posible redefinir nuevas componentes para el campo $A'_{\mu}$:}

\begin{align} \label{eq:spin1massless:cambiobase}
	C'_{R} &= \frac{1}{\sqrt{2}} \left( C'_{1} + iC'_{2} \right) 
	\hspace{17mm}
	C''_{R} = \frac{1}{\sqrt{2}} \left( C''_{1} + iC''_{2} \right) = e^{i\theta} C'_{R} \nonumber
	\\ &\hspace{32mm} \iff  \nonumber \\
	C'_{L} &= \frac{1}{\sqrt{2}} \left( C'_{1} - iC'_{2} \right) 
	\hspace{17mm}
	C''_{L} = \frac{1}{\sqrt{2}} \left( C''_{1} - iC''_{2} \right) = e^{-i\theta} C'_{R}.
\end{align}

\justify{Estas nuevas componentes, denominadas polarizaciones circulares, reflejan las helicidades del sistema. En esta base, las helicidades de las componentes, $C'_{R}$ y $C'_{L}$, son apreciables, presentando helicidades de $1$ y $-1$, respectivamente.}

\justify{Los resultados obtenidos indican que las helicidades del campo de spin-1 son consistentes con las predicciones de la representación de spin. Asimismo, se encuentran distribuidas en conformidad con la clasificación de Wigner, con componentes de helicidad $\pm1$ asociadas a los grados de libertad del sistema. La coherencia en el comportamiento de las helicidades de estos campos avala la conclusión del análisis del lagrangiano del spin-1 sin masa.}

\justify{Es relevante destacar no solo el análisis teórico, si no la correlación con lo observado fenomenológicamente: como por ejemplo el campo electromagnético, que se ajusta a este patrón. La helicidad de los fotones ha sido extensamente estudiada, lo cual nos permite distinguir dos subtipos de partículas en función de su helicidad: fotones levógiros y fotones dextrógiros (consulte \cite{Yang:2020eow} para más información).}

\justify{En conclusión, el estudio de los campos de spin-1 es crucial para comprender el comportamiento de estas partículas y sus interacciones con otros campos. Los campos de spin-1 masivos y los de spin-1 no masivos exhiben comportamientos distintos y requieren enfoques diferentes para su estudio, aunque se encuentran estrechamente relacionados. Los resultados obtenidos en ambos análisis respaldan la consistencia y previsibilidad de estos campos, proporcionando información valiosa sobre su estructura y propiedades.}