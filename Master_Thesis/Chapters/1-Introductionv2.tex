\setcounter{page}{1}
\pagestyle{plain}
\chapter{Introducción. } \label{Introduccion}

% 1 state the general topic and give some background:

\justify{Michael Faraday (1791-1869) introdujo innovadoramente el concepto de \textit{campo} en la física. En lugar de concebir las interacciones magnéticas como fenómenos que suceden instantáneamente a distancia, Faraday postuló la existencia de un ente mediador que transfiere información entre los elementos interactuantes. En este marco conceptual, el emisor origina una perturbación en su campo circundante, siendo la propagación de dicha perturbación a través del campo lo que eventualmente alcanza los receptores.}
%\justify{Michael Faraday (1791-1869) introdujo por primera vez el concepto de \textit{campo} en el ámbito de la física. En lugar de conceptualizar las interacciones magnéticas como si se produjeran a distancia, Faraday propuso la existencia de un mediador que transportaba información entre los objetos que interactuaban. Bajo este planteamiento, el emisor ejerce una perturbación sobre el campo de su entorno, y es la transmisión de esa perturbación a lo largo del campo la que llega a los receptores.}


\justify{La teoría clásica de campos, desarrollada entre finales del siglo XIX y principios del XX, impulsó significativamente nuestro entendimiento de la física. Este paradigma posibilita la modelización de las interacciones entre objetos a través de campos, conduciendo al desarrollo de la teoría del electromagnetismo en 1885 por James Clerk Maxwell (1831-1879) y, posteriormente en 1915, a la formulación de la teoría de la relatividad general por Albert Einstein (1879-1955). En términos generales, el concepto de campo ha sido un componente esencial en el progreso de la física moderna.}
%\justify{La teoría clásica de campos, desarrollada a finales del siglo XIX y principios del XX, hizo avanzar enormemente nuestra comprensión de la física. Este paradigma permite modelizar las interacciones entre objetos a través de campos, lo que condujo al desarrollo de la teoría del electromagnetismo en 1885 por James Clerk Maxwell (1831-1879) y, en 1915, de la teoría de la relatividad general por Albert Einstein (1879-1955). En general, el concepto de campo ha sido una parte crucial del desarrollo de la física moderna.}

\justify{El spin es una característica inherente de los campos en el contexto de la relatividad especial, y la teoría de campos de spin es el paradigma que encapsula el estudio de este tipo de campos. Dicha teoría constituye un marco crucial para la modelización de sistemas físicos. La teoría cuántica de campos profundiza en esta conceptualización al aplicar las reglas de cuantización a los campos descritos por la teoría de campos de spin, permitiendo así describir el comportamiento de los campos cuánticos.}
%\justify{El spin es una propiedad intrínseca de los campos en el marco de la relatividad especial, y la teoría de campos de spin es el paradigma que engloba el estudio de este tipo de campos. Esta teoría es un marco fundamental para modelar sistemas físicos. La teoría cuántica de campos da un paso más al aplicar reglas de cuantización a los campos descritos por la teoría de campos de spin para describir el comportamiento de los campos cuánticos.}


% 2 provide a review of the literature related to the topic

% 3 define the terms and scope of the topic

% 4 outline the current situation
\justify{La teoría de campos de spin, en conjunto con la teoría cuántica de campos, proporciona un marco eficaz para la modelización del comportamiento de las partículas fundamentales. Dichos campos cuánticos se categorizan principalmente en dos tipos: los campos fermiónicos, constitutivos de la materia, los cuales exhiben un spin semi-entero; y los campos bosónicos, mediadores de interacciones, que presentan un spin entero. Cabe destacar que todas las interacciones, a excepción de la gravedad, están incorporadas en el modelo estándar de partículas y son precisamente representadas por los campos bosónicos. Este modelo nos ofrece una descripción detallada del funcionamiento de las fuerzas fundamentales y la interacción entre todas las partículas elementales, incluyendo los campos bosónicos de spin 0 \cite{HiggsParticle}, spin 1 no masivo \cite{Aste:1998iw, Schwartz:2014sze}, y spin 1 masivo \cite{WZParticle, Aste:1997rh}.}
%\justify{En la práctica, la teoría de campos de spin y la teoría cuántica de campos se utilizan para modelar el comportamiento de las partículas fundamentales. Los campos cuánticos se clasifican en dos tipos: campos fermiónicos, que componen la materia y tienen spin semientero, y campos bosónicos, que median en las interacciones y presentan spin entero. Todas las interacciones, excepto la gravedad, forman parte del modelo estándar de partículas y se encuentran bien descritas por los campos bosónicos. Este modelo describe cómo funcionan las fuerzas y cómo interactúan todas las partículas elementales, e incluye los campos bosónicos de spin 0 \cite{HiggsParticle}, spin 1 no masivo \cite{Aste:1998iw, Schwartz:2014sze}, y spin 1 masivo \cite{WZParticle, Aste:1997rh}.}


\justify{La gravedad, que representa la única interacción no incorporada en el modelo estándar de física de partículas, se define primordialmente a través de la teoría de la relatividad general. Esta teoría se considerarse como una reinterpretación profunda de la gravedad newtoniana, al integrar la interacción gravitatoria dentro del marco de la teoría de la relatividad especial. En el contexto linealizado, la relatividad general muestra una similitud notable con el campo de spin-2, tal como se detalla en \cite{Wald:1984rg,JanssenBook}, propuesto y descrito por Markus Eduard Fierz (1912 - 2006) y Wolfgang Ernst Pauli (1900 - 1958) en el año 1939 \cite{Fierz:1939ix}. Este descubrimiento ha sido la base para la exploración y desarrollo de una teoría de la gravedad que se fundamenta en los campos de spin, con la posibilidad de conducir a una teoría cuántica de la gravedad.}
%\justify{La gravedad, la única interacción que no está contemplada en el modelo estándar de partículas, se describe actualmente mediante la teoría de la relatividad general. Esta teoría puede verse como una reformulación de la gravedad newtoniana que integra la interacción gravitatoria en el marco de la relatividad especial. En el régimen lineal, la relatividad general se asemeja al campo de spin-2 \cite{Wald:1984rg,JanssenBook} descrito por Markus Eduard Fierz (1912 - 2006) y Wolfgang Ernst Pauli (1900 - 1958) en 1939 \cite{Fierz:1939ix}. Este hallazgo ha motivado la exploración de una teoría de la gravedad fundamentada en campos de spin, y potencialmente, una teoría de la gravedad cuántica.}


% 5 evaluate the current situation (advantages/ disadvantages) and identify the gap
\justify{Las teorías de campos de spin se presentan como un marco formidable para desvelar los fundamentos de la física contemporánea, permitiendo adentrarnos en la esencia de las interacciones más básicas. Estas teorías son también indispensables en nuestra continua búsqueda de respuestas a cuestionamientos aún sin resolver en la física teórica. Investigaciones \textit{state-of-the-art} en el ámbito de nueva física, incluyendo la búsqueda de una partícula mediadora de la gravedad \cite{Wellmann:2001sx}, la exploración de física más allá del modelo estándar, que podría incluir posibles partículas constituyentes de la materia oscura \cite{Feng:2010gw, Cebrian:2022brv}, y el estudio del origen de la energía oscura \cite{Tsujikawa:2013fta}, son claros testimonios de su relevancia. Adicionalmente, las teorías de campos de spin han probado ser una herramienta crucial en el progreso de nuestro entendimiento de las fuerzas y partículas fundamentales de la naturaleza, tal como se puede apreciar en el descubrimiento del bosón de Higgs \cite{ATLAS:2012yve} y en la confirmación de la existencia de ondas gravitacionales.}
%\justify{Las teorías de campos de spin ofrecen un marco valioso para la exploración de los conceptos fundamentales de la física moderna y la comprensión de las interacciones más elementales. Estas teorías también juegan un papel esencial en la búsqueda de respuestas a las interrogantes actuales de la física teórica. Algunos ejemplos destacados de la investigación en nueva física incluyen la búsqueda de una partícula mediadora de la gravedad \cite{Wellmann:2001sx}, la exploración de física más allá del modelo estándar, como las partículas candidatas a constituir la materia oscura \cite{Feng:2010gw, Cebrian:2022brv}, y el estudio del origen de la energía oscura \cite{Tsujikawa:2013fta}. Además, las teorías de campos de spin han demostrado ser fundamentales para avanzar en nuestra comprensión de las fuerzas y partículas fundamentales de la naturaleza, como se evidencia en el descubrimiento del bosón de Higgs \cite{ATLAS:2012yve} y la confirmación de la existencia de ondas gravitacionales}


% 6 identify the importance of the proposed research

\justify{La investigación en la teoría de campos de spin resulta de primordial relevancia en el dominio de la física de interacciones fundamentales. Esta tesis aspira a ofrecer un análisis fundamental de las teorías de campo bosónico, poniendo énfasis en la adquisición y desarrollo de habilidades fundamentales para la construcción de dichas teorías. Con la inmersión en este tema, se busca profundizar nuestra comprensión de área de estudio y aportar contribuciones significativas en nuestro entendimiento de las interacciones fundamentales.}
%\justify{El estudio de la teoría de campos de spin es de suma importancia en el campo de la física de interacciones fundamentales. Esta tesis tiene como objetivo proporcionar un análisis exhaustivo de las teorías de campo bosónico, con un enfoque en el desarrollo de habilidades esenciales en la construcción de tales teorías. A través de la exploración de este tema, pretendemos obtener una comprensión más profunda del estado actual del conocimiento en el campo y hacer contribuciones significativas en nuestra comprensión de las interacciones fundamentales.}

% 7 state the research problem/ questions

% 8 state the research aims and/or research objectives



% 9 state the hypotheses

% 10 outline the order of information in the thesis

\justify{La estructura de la presente tesis es la siguiente: En el Capítulo \ref{Preliminar}, introducimos tres elementos conceptuales clave que son indispensables para el análisis de las teorías que se abordarán en los capítulos subsiguientes. Estos elementos son el teorema de Ostrogradsky, la clasificación de Wigner y el dual de Hodge. El teorema de Ostrogradsky nos brinda una herramienta para identificar posibles desviaciones anómalas de un modelo físico en comparación con las expectativas clásicas. La clasificación de Wigner nos proporciona un método para prever los grados de libertad de un campo de spin. Por otro lado, el dual de Hodge se trata de un operador matemático, este será utilizado en el análisis del campo de Kalb-Ramond. Estos conceptos constituyen pilares fundamentales para nuestro examen de las teorías expuestas en los capítulos venideros.}
%\justify{La estructura de esta tesis es la siguiente: En el capítulo \ref{Preliminar}, introducimos tres conceptos clave que resultan relevantes para el análisis de las teorías presentadas en los capítulos siguientes. Estos conceptos son el teorema de Ostrogradsky, la clasificación de Wigner y el dual de Hodge. El teorema de Ostrogradsky nos permite identificar posibles comportamientos anómalos de un modelo físico con respecto a las predicciones clásicas. La clasificación de Wigner permite predecir los grados de libertad de un campo de spin. El dual de Hodge, por su parte, es un operador matemático que utilizaremos en el análisis del campo de Kalb-Ramond. Estos conceptos serán esenciales para nuestro estudio de las teorías presentadas en los capítulos siguientes.}

\justify{El objetivo de los capítulos sucesivos es elaborar y analizar los lagrangianos de varios campos bosónicos. En particular, nos enfocamos en tres categorías de campos tanto en sus formas masivas como no masivas. En principio, tras introducir los antecedentes necesarios en el segundo capítulo, nos enfocaremos en estudiar los campos de spin-1 con mayor profundidad.}
%\justify{El objetivo de los siguientes capítulos es desarrollar y analizar los lagrangianos de varios campos bosónicos. En particular, nos centramos en tres tipos de campos tanto en sus formas masivas como no masivas. En primer lugar, tras introducir los antecedentes necesarios en el segundo capítulo, procedemos a estudiar los campos de spin-1 con más detalle.}

\justify{En el Capítulo \ref{Spin-1Massive}, examinamos el caso masivo del campo de spin-1, donde el campo se encuentra asociado a una masa no nula. Este capítulo nos brinda la oportunidad de explorar el comportamiento del campo desde la perspectiva más general y deducir que el lagrangiano de Proca es el único lagrangiano físicamente viable. En el Capítulo \ref{Spin-1Massless}, desviamos nuestra atención hacia el caso sin masa, el cual se distingue por la ausencia de un término de masa en el lagrangiano del campo. Al igual que en el caso masivo, deduciremos que el único lagrangiano físicamente aceptable es el lagrangiano de Maxwell. Adicionalmente, investigaremos las simetrías y leyes de conservación que emergen en el caso no masivo, y cotejaremos el comportamiento de los campos de spin-1 masivos y sin masa.}
%\justify{En el capítulo \ref{Spin-1Massive}, consideramos el caso masivo del campo de spin-1, en este caso el campo se encuentra asociado a una masa distinta de cero. Este capítulo nos permite explorar el comportamiento del campo desde la perspectiva más general y deducir el lagrangiano de Proca como el único lagrangiano físicamente posible. En el capítulo \ref{Spin-1Massless}, dirigimos nuestra atención al caso sin masa, que se caracteriza por la ausencia de un término de masa en el lagrangiano del campo. Al igual que ocurre con el caso masivo, deduciremos que el único lagrangiano físicamente aceptado es el lagrangiano de Maxwell. Además, examinamos las simetrías y leyes de conservación que surgen en el caso no masivo, y comparamos el comportamiento de los campos sin masa y masivos de spin-1.}

\justify{En conjunto, estos capítulos proporcionan una comprensión integral de los campos de spin-1 y sus propiedades. Dado que estos dos casos son sencillos y bien conocidos en la comunidad científica, los utilizamos como punto de partida para introducir los conceptos fundamentales de la teoría de campos.}
%\justify{En conjunto, estos capítulos proporcionan una comprensión completa de los campos de spin-1 y sus propiedades. Estos dos casos son sencillos y bien conocidos por la comunidad, por este motivo los utilizamos como punto de partida para introducir los conceptos del paradigma de la teoría de campos.}

\justify{En segundo lugar, nos enfocaremos en los campos de spin-2 masivos y no masivos (Capítulos \ref{spin-2Masivo} y \ref{spin-2Sinmasa}), los cuales, aunque más complejos, comparten numerosas similitudes con los campos de spin-1. Estos campos son de particular interés porque pueden utilizarse para describir las interacciones entre partículas con spin-2, como los teóricos gravitones. Adicionalmente, su comportamiento es esencial para entender las propiedades de ciertas teorías vinculadas con la gravedad, tales como la relatividad general, la gravedad masiva o incluso la teoría de cuerdas.}
%\justify{En segundo lugar, nos centraremos en los campos masivos y no masivos de spin-2 (capítulos \ref{spin-2Masivo} y \ref{spin-2Sinmasa}), que son más complejos pero comparten muchas similitudes con los campos de spin-1. Estos campos resultan interesantes de estudiar porque pueden utilizarse para describir las interacciones entre partículas con spin-2, como los teóricos gravitones. Además, su comportamiento es crucial para comprender las propiedades de ciertas teorías realacionadas con la graverdad, como la relatividad general, la gravedad masiva o incluso la teoría de cuerdas.}

\justify{Finalmente, en nuestra investigación, nos adentramos en el estudio de los campos de Kalb-Ramond, que representan los ejemplos más complejos y abstractos abordados en esta tesis (Capítulos \ref{Kalb-Ramond-massive} y \ref{Kalb-Ramond-massless}). En concreto, el estudio de los campos de Kalb-Ramond ilumina sobre las propiedades fundamentales de la teoría de cuerdas y sus aplicaciones en física. De manera específica, haremos uso de la relación entre los campos de Kalb-Ramond y el campo de spin 1, así como la conexión entre el spin y la naturaleza masiva o sin masa de estos campos.}
%\justify{En el análisis final, nos adentramos en la investigación de los campos de Kalb-Ramond, que son los ejemplos más complejos y abstractos considerados en esta tesis (capítulos \ref{Kalb-Ramond-massive} y \ref{Kalb-Ramond-massless}). En particular, el estudio de los campos de Kalb-Ramond arroja luz sobre las propiedades fundamentales de la teoría de cuerdas y sus aplicaciones en física. En concreto, descubrimos la relación entre los campos de Kalb-Ramond y otros campos conocidos, así como la conexión entre el spin y la naturaleza masiva o sin masa de estos campos.}


% 11 outline the methodology

\justify{La metodología empleada en el análisis de cada teoría es similar. La tesis presta especial atención al estudio de los grados de libertad de los campos asociados a las teorías, al examen de los grados de libertad aparentes, es decir no físicos, y a la búsqueda de restricciones que conduzcan al número de grados de libertad predicho por la clasificación de Wigner.}


\justify{En cada capítulo derivamos la ecuación de movimiento para el sistema físico utilizando la ecuación de Euler-Lagrange y procedemos a realizar la transformada de Legendre para obtener el hamiltoniano de la teoría. A continuación, demostramos que el dominio del hamiltoniano se encuentra acotado con un límite inferio, lo que garantiza que el sistema no presenta inestabilidades de Ostrogradsky. Este último resultado es crucial para la estabilidad y predictibilidad del sistema.}


\justify{Por último, analizamos el comportamiento del spin de cada componente individual de los campos no masivos. Resulta esencial para este estudio confirmar que los campos exhiben el spin esperado, así como identificar qué componentes corresponden a grados de libertad del campo y los valores de spin asociados.}


\justify{Esta metodología nos permite analizar todos los sistemas físicos de cada capítulo, particularmente en tres aspectos: investigaremos la relación entre el carácter tensorial y el spin de un campo. Examinaremos los casos masivos y sin masa, comparando y contrastando sus diferencias y similitudes. Además, exploraremos cómo la conservación de una corriente que interactúa con el campo está relacionada con la simetría del sistema, y cómo estas simetrías pueden utilizarse para predecir la conservación de la carga en casos no masivos. Específicamente, investigaremos la conexión entre la conservación de una corriente física  en los casos no masivos y las simetrías del sistema que este conlleva. }

\newpage

\section*{\centering Notación y Convenios Utilizados}

\justify{Con el propósito de establecer un marco interpretativo coherente, y facilitar la comprensión del contenido presentado en esta tesis, se adoptan ciertos convenios y notaciones convencionales en el ámbito de la física teórica.}

\justify{Inicialmente, utilizamos el sistema de unidades naturales, en el cual la velocidad de la luz se establece en la unidad ($c = 1$). Por otro lado, se emplea la letra $D$ para denotar el número de dimensiones de la variedad manifiesta; no obstante, gran parte de la tesis se centra en el caso específico de cuatro dimensiones ($D = 4$).}

\justify{Adoptamos el convenio de signos de tiempo positivo para la métrica $(+, -, -, -)$. En relación con la notación de los índices, los índices griegos ($\alpha$, $\beta$, $\gamma$ ...) se utilizan para denotar índices espacio-temporales ($\alpha = 0, 1, \cdots, D$), mientras que los índices latinos ($i$, $j$, $k$ ...) se reservan para representar exclusivamente las componentes espaciales ($i = 1, \cdots, D-1$). El valor 0 de los índices hace referencia a la dirección temporal.}

\justify{Además, se adopta el convenio de simetrización y antisimetrización: $(a, b) = ab + ba$ denota la simetrización, y $[a, b] = ab - ba$ representa la antisimetrización.}

\justify{En el contexto de los productos tensoriales, hay circunstancias en las que resulta crucial distinguir de manera explícita entre las componentes temporales y espaciales. A modo de ejemplo:}

\begin{align}
	\Pi^{00i}\Pi_{00i} &=  \eta^{il} \; \Pi_{00i}\Pi_{00m} = - \delta^{il} \; \Pi_{00i}\Pi_{00m}, \\
	\Pi^{0ij}\Pi_{0ij}   &=  \eta^{il}\eta^{jm} \; \Pi_{0ij}\Pi_{0lm} = \delta^{il}\delta^{jm} \; \Pi_{0ij}\Pi_{0lm}, \\
	\Pi^{ijk}\Pi_{ijk}   &=  \eta^{il}\eta^{jm} \eta^{kn} \; \Pi_{ijk}\Pi_{lmn} = - \delta^{il}\delta^{jm} \delta^{kn}\; \Pi_{ijk}\Pi_{lmn}.
\end{align}

\justify{En este contexto, $\delta^{ij}$ representa el símbolo de Kronecker. Para el desarrollo de la tesis, es fundamental establecer una notación precisa, por ejemplo $\left(\Pi_{0ij}\right)^{2}$, para manejar la parte estrictamente positiva de cada término, independientemente de los signos asociados a la métrica. Los casos anteriores se manejarían como sigue de acuerdo con este convenio:}

\begin{align}
	(\Pi_{00i})^{2} &\equiv \delta^{il} \; \Pi_{00i}\Pi_{00m}, \\
	(\Pi_{0ij})^{2}  &\equiv \delta^{il}\delta^{jm} \; \Pi_{0ij}\Pi_{0lm}, \\
	(\Pi_{ijk})^{2}    &\equiv \delta^{il}\delta^{jm} \delta^{kn}\; \Pi_{ijk}\Pi_{lmn}.
\end{align}

\justify{Los convenios y notaciones anteriores serán utilizados de forma sistemática a lo largo de este trabajo, lo que garantiza una exposición del contenido que sea clara, precisa y académicamente rigurosa.}

